\documentclass{article}
\usepackage{multirow}
\usepackage{array}
\usepackage{tabularx}
\usepackage{hyperref}
\usepackage{colortbl}
\usepackage{xcolor}
\usepackage{amssymb}
\usepackage{float}
\usepackage{hyperref}
\usepackage{booktabs}
\usepackage{soul}
\usepackage[version=4]{mhchem}
\usepackage[legalpaper, portrait, margin=1in]{geometry}
\begin{document}

\section*{TODOp}
\begin{enumerate}
    \item Find how garrett made charts in his manuscript 
    \item Make charts better or change input data
    \item Do comparisons of Cl species with F species
    \begin{enumerate}
        \item Garrett seems to have started doing that in lagrangian hi res mass frac.ipynb, but he doesn't include measured Cl data.
    \end{enumerate}
\end{enumerate}

\subsubsection*{cfs2canteraPD}
\begin{enumerate}
    \item Get injected\_species from netj\_file
    \item Get conc\_cutoff from netj\_file (inject conc)
    \item Make it run PFRTP-Garrett and cfs\_post if needed
\end{enumerate}

\subsubsection*{cfs\_post}
\begin{enumerate}
    \item Make trajectory selection faster
    \item Make it run PFRTP-Garrett if needed
\end{enumerate}

\subsubsection*{PFRTP-Garrett}
\begin{enumerate}
    \item Make interactive widget to choose run conditions
    \item Fix thing with flame temperature estimate
    \item Temperatures shouldn't go down then back up
\end{enumerate}

\section*{Instructions}
\begin{enumerate}
    \item Run \href{netj_generate.ipynb}{netj\_generate.ipynb} to generate input files from lists of run conditions
    \item Run \href{cfs_batchrun.ipynb}{cfs\_batchrun.ipynb} to send the generated .netj files to CFS and run the simulations
    \item Run \href{PFRTP-Garrett.ipynb}{PFRTP-Garrett.ipynb} to use Cantera to perform the same simulations.
    \item Run \href{cfs_post.ipynb}{cfs\_post.ipynb}, and select .netj files.
    \item Run \href{cfs2canteraPD.ipynb}{cfs2canteraPD.ipynb} to generate path diagrams using Cantera.
\end{enumerate}

\section*{Problems}
\begin{enumerate}
    \item There are many many .ipynb files in Modeling\_GPD, many of which are near duplicates of each other, scattered across folders. It is difficult to tell which ones are the most up to date and which ones are obsolete.
    \item Residence Time column from cantera has no duplicate times. CFS will have 3-7 rows of identical times
    \item Cantera increases time steps constantly, CFS has variable time steps
    \item Concentrations of non-major species in cantera are around 1e-100, in CFS they are around 1e-15
    \begin{enumerate}
        \item Produces problems for reaction path diagrams because cantera can’t tell which is the major species
        \item Concentrations are all 0 until injection, then they all instantly increase
        \begin{enumerate}
            \item This is probably because the transient solution solves from the injection point and then tries to work backwards to the flame. Injecting at the flame may fix this.
        \end{enumerate}
    \end{enumerate}
    \item CF3Or has ~15 reactions, CF2s has ~74, CF4 ~15, C2F6 ~6. C2F6 behavior in path diagrams may be caused by incomplete reactions
    \begin{enumerate}
        \item C2F6 may just not have that many reactions because it is a larger molecule, and the C-C bond is the most likely to break
    \end{enumerate}
    \item Several reactions are “ambiguous” and cantera will use the designated default reaction
    \begin{enumerate}
        \item Default reaction may not be conducive for simulating C2F6
    \end{enumerate}
    \item Some reactions are not parsed (they may just not be relevant)
\end{enumerate}
\section*{Notes}
\begin{enumerate}
    \item CF4 will never be destroyed at temps below 1200K, 100\% DE for temps >1600 C
    \item Higher values of MAXIT seem to produce more residence time steps
\end{enumerate}
\pagebreak
\begin{table}[H] %files, paths, and descriptions
\centering
\renewcommand{\arraystretch}{1.256}
\caption{Files, Folders, and Descriptions}
\begin{tabular}{>{\centering\arraybackslash}m{.28\textwidth}>{\scriptsize\centering\arraybackslash}m{0.3\textwidth}>{\footnotesize\centering\arraybackslash}m{0.28\textwidth}}\toprule

\normalsize{\textbf{File}}      & \normalsize{\textbf{Folder}}      & \normalsize{\textbf{Description}} \\\cmidrule(r){1-1}\cmidrule(lr){2-2}\cmidrule(l){3-3}

flow\_convert.ipynb         & PFAS\_Modeling/       & Converts liters per minute to kilograms per second for air and for CH4 using cantera \\
netj\_generate.ipynb        & PFAS\_Modeling/   & Takes a set of varying run conditions and produces netj files to do all iterations \\
cfs\_batchrun.ipynb         & PFAS\_Modeling/       & Uses CFS batch method to run generated netj files, prompts user for inputs to determine simulation duration \\
PFRTP-Garrett.ipynb         & PFAS\_Modeling/Pseudo\_PFR        & Uses a PFR simulation in Cantera to make similar results as CFS \\
cfs\_post.ipynb         & PFAS\_Modeling/       & Runs CFS in GUI mode to automate post-processing of vtk files to include species concentration \\
cfs2canteraPD.ipynb         & PFAS\_Modeling/       & Converts vtk to csv data, then produces a path diagram using Cantera for a specified trajectory, residence time, and threshold. \\
pfr.ipynb       & PFAS\_Modeling/       & Example PFR that came with Cantera \\
pfr2.ipynb      & PFAS\_Modeling/       & Example Cantera PFR that includes heat loss \\
vtk2csv.py      & PFAS\_Modeling/       & Stand alone script for converting a specific vtk to a csv \\
lagrangian hi res mass frac.ipynb       & L:/Lab/AMCD\_PFAS\_Incineration/ Modeling\_GPD/Cantera/PFR DE S Curves/       & Makes plots of multiple species DE\% vs Temperature \\
eulerian \(<\)injectant\(>\).ipynb      & L:/Lab/AMCD\_PFAS\_Incineration/ Modeling\_GPD/Cantera/influent\ concentration/       & Make multidimensional color plots for conc. vs Temperature vs DE\%. Also absolute and relative yield vs conc. \\
paths \(<\)SR\(>\).ipynb        & L:/Lab/AMCD\_PFAS\_Incineration/ Modeling\_GPD/Cantera/paths/     & Makes path diagrams for several species at specified SR \\
T99.ipynb       & L:/Lab/AMCD\_PFAS\_Incineration/ Modeling\_GPD/Cantera/T99/       & Makes T99 plots of multiple species \\
lagrangian.ipynb        & L:/Lab/AMCD\_PFAS\_Incineration/ Modeling\_GPD/Cantera/PFR\ DE\ S\ Curves/archive/    & Makes plots of species DE vs Temperature. Plots from this script don't look the same as plots in the doc, so it is probably outdated. \\
\bottomrule
\end{tabular}
\end{table}

\pagebreak
\begin{table}[H] %cfs valid diagrams
\centering
\caption{Validated Path Diagrams - CFS}
\begin{tabular}{|c|c|c|c|c|c|c|c|c|c|c|c|c|}
\hline
        & \multicolumn{4}{c|}{CF4}      & \multicolumn{4}{c|}{CHF3}     & \multicolumn{4}{c|}{C2F6} \\
\hline
        & \multicolumn{4}{c|}{Ports}        & \multicolumn{4}{c|}{Ports}    & \multicolumn{4}{c|}{Ports} \\
\hline
        & 1         & 4         & 6         & 8         & 1         & 4         & 6         & 8         & 1         & 4         & 6         & 8 \\
\hline
27.5 kW 
        &                   %CF4 port 1
        &                   %CF4 port 4
        &                   %CF4 port 6
        &                   %CF4 port 8
        &                   %CHF3 port 1
        &                   %CHF3 port 4
        &                   %CHF3 port 6
        &                   %CHF3 port 8
        &    \texttimes     %C2F6 port 1
        &                   %C2F6 port 4
        &                   %C2F6 port 6
        &    \texttimes \\  %C2F6 port 8
\hline
45 kW 
        &   \texttimes      %CF4 port 1
        &   \checkmark      %CF4 port 4
        &                   %CF4 port 6
        &                   %CF4 port 8
        &                   %CHF3 port 1
        &                   %CHF3 port 4
        &                   %CHF3 port 6
        &                   %CHF3 port 8
        &  \texttimes       %C2F6 port 1
        &  \texttimes       %C2F6 port 4
        &                   %C2F6 port 6
        &  \texttimes   \\  %C2F6 port 8
\hline
\end{tabular}
\end{table}

\begin{table}[H] %cantera valid diagrams
\centering
\caption{Validated Path Diagrams - Cantera}
\begin{tabular}{|c|c|c|c|c|c|c|c|c|c|c|c|c|}
\hline
        & \multicolumn{4}{c|}{CF4}      & \multicolumn{4}{c|}{CHF3}     & \multicolumn{4}{c|}{C2F6} \\
\hline
        & \multicolumn{4}{c|}{Ports}        & \multicolumn{4}{c|}{Ports}    & \multicolumn{4}{c|}{Ports} \\
\hline
        & 1         & 4         & 6         & 8         & 1         & 4         & 6         & 8         & 1         & 4         & 6         & 8 \\
\hline
27.5 kW 
        &\texttimes         %CF4 port 1
        &\texttimes         %CF4 port 4
        &\texttimes         %CF4 port 6
        &\texttimes         %CF4 port 8
        &                   %CHF3 port 1
        &                   %CHF3 port 4
        &                   %CHF3 port 6
        &                   %CHF3 port 8
        &\checkmark         %C2F6 port 1
        &\checkmark         %C2F6 port 4
        &\checkmark         %C2F6 port 6
        &\checkmark     \\  %C2F6 port 8
\hline
45 kW 
        &\checkmark         %CF4 port 1
        &\checkmark         %CF4 port 4
        &\checkmark         %CF4 port 6
        &\checkmark         %CF4 port 8
        &                   %CHF3 port 1
        &                   %CHF3 port 4
        &                   %CHF3 port 6
        &                   %CHF3 port 8
        &\checkmark         %C2F6 port 1
        &\checkmark         %C2F6 port 4
        &\checkmark         %C2F6 port 6
        &\checkmark     \\  %C2F6 port 8
\hline
\end{tabular}
\end{table}

\begin{table}[H] %cfs executed simulations
\centering
\caption{CFS Executed Simulations}
\begin{tabular}{|c|c|c|c|c|c|c|c|c|c|c|c|c|}
\hline
        & \multicolumn{4}{c|}{CF4}      & \multicolumn{4}{c|}{CHF3}     & \multicolumn{4}{c|}{C2F6} \\
\hline
        & \multicolumn{4}{c|}{Ports}        & \multicolumn{4}{c|}{Ports}    & \multicolumn{4}{c|}{Ports} \\
\hline
        & 1         & 4         & 6         & 8         & 1         & 4         & 6         & 8         & 1         & 4         & 6         & 8 \\
\hline
27.5 kW 
        &                   %CF4 port 1
        &\checkmark         %CF4 port 4
        &                   %CF4 port 6
        &                   %CF4 port 8
        &                   %CHF3 port 1
        &                   %CHF3 port 4
        &                   %CHF3 port 6
        &                   %CHF3 port 8
        &\checkmark         %C2F6 port 1
        &\checkmark         %C2F6 port 4
        &\checkmark         %C2F6 port 6
        &\checkmark     \\  %C2F6 port 8
\hline
45 kW 
        &\checkmark         %CF4 port 1
        &\checkmark         %CF4 port 4
        &                   %CF4 port 6
        &                   %CF4 port 8
        &\checkmark         %CHF3 port 1
        &\checkmark         %CHF3 port 4
        &\checkmark         %CHF3 port 6
        &                   %CHF3 port 8
        &\checkmark         %C2F6 port 1
        &\checkmark         %C2F6 port 4
        &\checkmark         %C2F6 port 6
        &\checkmark     \\  %C2F6 port 8
\hline
\end{tabular}
\end{table}

\begin{table}[H] %cantera executed simulations
\centering
\caption{Cantera Executed Simulations}
\begin{tabular}{|c|c|c|c|c|c|c|c|c|c|c|c|c|}
\hline
        & \multicolumn{4}{c|}{CF4}      & \multicolumn{4}{c|}{CHF3}     & \multicolumn{4}{c|}{C2F6} \\
\hline
        & \multicolumn{4}{c|}{Ports}        & \multicolumn{4}{c|}{Ports}    & \multicolumn{4}{c|}{Ports} \\
\hline
        & 1         & 4         & 6         & 8         & 1         & 4         & 6         & 8         & 1         & 4         & 6         & 8 \\
\hline
27.5 kW 
        &\checkmark         %CF4 port 1
        &\checkmark         %CF4 port 4
        &\checkmark         %CF4 port 6
        &\checkmark         %CF4 port 8
        &                   %CHF3 port 1
        &                   %CHF3 port 4
        &                   %CHF3 port 6
        &                   %CHF3 port 8
        &\checkmark         %C2F6 port 1
        &\checkmark         %C2F6 port 4
        &\checkmark         %C2F6 port 6
        &\checkmark     \\  %C2F6 port 8
\hline
45 kW 
        &\checkmark         %CF4 port 1
        &\checkmark         %CF4 port 4
        &\checkmark         %CF4 port 6
        &\checkmark         %CF4 port 8
        &                   %CHF3 port 1
        &                   %CHF3 port 4
        &                   %CHF3 port 6
        &                   %CHF3 port 8
        &\checkmark         %C2F6 port 1
        &\checkmark         %C2F6 port 4
        &\checkmark         %C2F6 port 6
        &\checkmark     \\  %C2F6 port 8
\hline
\end{tabular}
\end{table}

\section*{Notes from Bill}
% Kaleb:

% I have an idea on how to approach an introduction on the comparison of the thermal destruction of F and Cl species.  
% Back on the 1980s, EPA was mostly interested in chlorinated solvents and industrial important chlorinated compounds.  
% Prior to the use of industrial and commercial incinerators, these liquid wastes were commonly dumped into pits or buried in drums.  
% NCSU had their own Superfund site next to Carter-Finley.  
% Ocean dumping was also common.  
% Many of the Superfund sites that were or are still being cleaned up are the result of land disposal of these solvents.  
% Many chlorinated.

% EPA developed an Incinerability Index to rank the relative difficulty to thermally destroy a variety of compounds.  
% Over 300 species were evaluated experimentally.  
% I am attaching a report about this.  
% \hl{Table D-1 (page 109) lists the species by rank (1-320).
% Table D-2 (page 115) list these alphabetically.  }
% I believe that several fluorinated species are included (but not many).  
% Species were grouped into 7 Classes.  
% Class 1 most difficult to destroy. 
% Class 7 least difficult.

\begin{enumerate}
    \item go through these lists and identify any C1, C2, and possibly C3 F and Cl species and their ranking
    \item (CCl4 (\checkmark, 136-140), CHCl3(\checkmark, 195-196), C2Cl6(\checkmark, 202-203), CF4(\texttimes), CHF3(\texttimes), C2F6(\texttimes)) listed?
    \item other C1, C2, and C3 chloro or fluorocarbons
    \begin{table}
        \caption{C1, C2, and C3 Chlorinated and Fluorinated Species}
        \centering
        \footnotesize
        \begin{tabular}{>{\raggedright\arraybackslash}m{.1\textwidth}>{\raggedright\arraybackslash}m{.2\textwidth}>{\raggedright\centering\arraybackslash}m{.12\textwidth}|>{\raggedright\arraybackslash}m{.1\textwidth}>{\raggedright\arraybackslash}m{.2\textwidth}>{\raggedright\centering\arraybackslash}m{.12\textwidth}}\toprule
            Formula             & Species Name      & Incinerability Index  & Formula           & Species Name                  & Incinerability Index \\\cmidrule(r){1-1}\cmidrule(lr){2-3}\cmidrule(r){4-4}\cmidrule(r){5-6}
            \ce{CNCl}           & Cyanogen Chloride         & 17-18         & \ce{C3H6Cl2}      & 1,3-Dichloropropane           & 165       \\
            \ce{CH3Cl}          & Chloromethane             & 29-30         & \ce{C3H5Cl3}      & 1,2,3-Trichloropropane        & 168-173   \\
            \ce{COCl2}          & Phosgene                  & 39-40         & \ce{C2H4Cl2}      & 1,1-Dichloroethane            & 175-178   \\
            \ce{C2H3ClO2}       & Methyl Chloroformate      & 46-50         & \ce{C3H5ClO}      & 1-Chloro-2,3-epoxypropane     & 183-186   \\
            \ce{C2H2Cl2}        & Dichloroethene            & 54            & \ce{CHCl3S}       & Trichloromethanethiol         & 189-192   \\
            \ce{C2H4FNO}        & Fluoroacetamide           & 55-56         & \ce{C2Cl6}        & Hexachloroethane              & 202-203   \\
            \ce{C2H3Cl}         & Vinyl Chloride            & 60-64         & \ce{C2H5ClO}      & Chloromethyl Methyl Ether     & 218-220   \\
            \ce{CH2Cl2}         & Dichloromethane           & 65-66         & \ce{C2H4Cl2O}     & bis(Chloromethyl) Ether       & 222-223   \\
            \ce{C3H4Cl2}        & 1,2-Dichloropropene       & 89-91         & \ce{C3Cl6}        & Hexachloropropene             & 234       \\
            \ce{CH3COCl}        & Acetyl Chloride           & 92-97         &                   &                               &           \\
            \ce{C2H2Cl4}        & Tetrachloroethane         & 121-125       &                   &                               &           \\
            \ce{C2H5Cl}         & Chloroethane              & 126           &                   &                               &           \\
            \ce{C2H4Cl2}        & Dichloroethane            & 131           &                   &                               &           \\
            \ce{C3H4ClN}        & 3-Chloropropionitrile     & 143-144       &                   &                               &           \\
            \ce{C3H6Cl2O}       & 1,3-Dichloropropan-2-ol   & 145-146       &                   &                               &           \\
            \ce{CHClF2}         & Chlorodifluoromethane     & 151-153       &                   &                               &           \\
            \ce{CHCl2F}         & Dichlorofluoromethane     & 154-157       &                   &                               &           \\
            \ce{C2HCl5}         & Pentachloroethane         & 154-157       &                   &                               &           \\
            \ce{C2H3Cl3}        & Trichloroethane           & 158-161       &                   &                               &           \\
            \ce{CHCl3}          & Chloroform                & 158-161       &                   &                               &           \\\bottomrule
        \end{tabular}
    \end{table}
    \item mixed Cl-F species?
    \begin{itemize}
        \item mainly CFCs
    \end{itemize}
    \item analyze the fraction of chlorinated species, and number of fluorinated species included
    \begin{itemize}
        \item \hl{Does this mean \emph{all} chlorinated species, or only those with 1-3 C?}
        \item There are 320 total species in the list. 
        \item Chlorinated species account for 113 species. Chlorinated C1-C3 species account for 40 of them.
        \item Fluorinated species account for 8 species. Only twe do not contain 1 to 3 carbons (Sulfur Hexafluoride and Fluoroacetic Acid)
    \end{itemize}
    \item anything you think notable.
    \begin{itemize}
        \item It's interesting how the only fluorinated species on the list that aren't CFCs are \ce{SF6} and \ce{C2H3FO2}
    \end{itemize}
    \item Find incinerability index for species %will hopefully support selection
\end{enumerate}
 
I’d like to begin the introduction of our Cl/F paper with a discussion of the Incinerability Index, and why we chose to study these 6 compounds. The main reasons are their \hl{combinations of different molecular structures}, bond types, and their \hl{available/published kinetics}. %hopefully this will be supported by the incinerability index

\subsubsection*{Talk about why the incinerability index is used instead of other measures.}

\subsubsection*{Talk about properties of these compounds.} 

Heat of combustion was used for a while, but it wasnt accurate because: 

\textbf{Why, specifically, is the thermal stability used instead of other measures?}

\textbf{Why are these compounds chosen? What makes them good candidates for this study?}

    \textbf{What is it about the molecular structures that makes these representative compounds?}

        CCl4 has tetrahedral, single covalent bonds. CHCl3 is tetrahedral, like methane but with 3 Hs replaced with Cls. C2Cl6 is two carbons connected linearly, with each carbon bonded to three chlorine atoms in a trigonal planar arrangement. CF4 is tetrahedral, like methane but with all Hs replaced with Fs. CHF3 is tetrahedral, like methane but with 2 Hs replaced with Fs. C2F6 is two carbons connected linearly, with each carbon bonded to three fluorine atoms in a trigonal planar arrangement. The fluorine compounds all have higher bond energies than the chlorinated compounds. 

    \textbf{What is it about the bond types that makes these representative compounds?}

        Not sure if he means that having the fluorinated and chlorinated compounds allows for comparing the behavior of the bond energies of the C-F and C-Cl bonds, or if he means that the bond types of the compounds themselves are important.

        Having the chlorinated analogues of the fluorinated compounds could allow for establishing a routine to compare the two?

    \textbf{What is it about the kinetics that makes these representative compounds?}

        I think it's because the mechanisms of destruction of these compounds are expected to be used in more complex compounds. Basically, larger PFAS probably break apart into these compounds.

\subsubsection*{Make comparisons of fluorinated and chlorinated compounds}

    All the fluorinated compounds have higher bond energies, and usually slightly denser molecules.

As you will see many of the Class 1 species are PAHs with ring structures.  
\begin{itemize}
    \item Most of the PAHs have many carbons, so I'm not sure if they're relevant for the discussion of C1-C3 species
\end{itemize}

I’m thinking a discussion of these species and bond energies might be a good place to start.
%make discussion of bond energies and structures. By 'these species' does he mean the C1-C3 chlorinated and fluorinated compounds, the Class 1 Chlorinated PAHs, or all of the species in the incinerability index with Chlorine or Fluorine?
\end{document}
